%% LaTeX Beamer presentation template (requires beamer package)
%% see http://latex-beamer.sourceforge.net/
%% idea contributed by H. Turgut Uyar
%% template based on a template by Till Tantau
%% this template is still evolving - it might differ in future releases!

\documentclass[14pt]{beamer}

\mode<presentation>
%\mode<handout>
{
\usetheme{Berlin}

\setbeamercovered{transparent}
}

\usepackage[english]{babel}
\usepackage[latin1]{inputenc}


% font definitions, try \usepackage{ae} instead of the following
% three lines if you don't like this look
%%%\usepackage{mathptmx}
%%%\usepackage[scaled=.90]{helvet}
%%%\usepackage{courier}

\ifx\pdftexversion\undefined
\usepackage[dvips]{graphicx}
\else
\usepackage{graphicx}
\DeclareGraphicsRule{*}{mps}{*}{}
\fi

\usepackage{graphics}
%\usepackage{makeidx}
%\usepackage{mathpazo}
%\usepackage{multicol}
%\usepackage{srcltx}
\usepackage{color}
\usepackage{eurosym}
\usepackage{tikz}
\usepackage{pgfplots}
\pgfplotsset{every axis plot/.append style={very thick}}


\title{ISDN - Handlungsschritt eins}

%\subtitle{}

% - Use the \inst{?} command only if the authors have different
%   affiliation.
%\author{F.~Author\inst{1} \and S.~Another\inst{2}}
\author{Petra~Fronius \and Conrad~Kostecki \and Oluf~Lorenzen \and Christoph~Ringe \and Martin~Wilke}
% \inst{}

% - Use the \inst command only if there are several affiliations.
% - Keep it simple, no one is interested in your street address.
\institute
{
\inst \normalfont
Multi Media\\
\normalfont
Berufsbildende Schulen\\
}

\date{LF 9\\15. 12. 2008}


% This is only inserted into the PDF information catalog. Can be left
% out.
%%%\subject{Talks}



% If you have a file called "university-logo-filename.xxx", where xxx
% is a graphic format that can be processed by latex or pdflatex,
% resp., then you can add a logo as follows:

\pgfdeclareimage[height=0.5cm]{mmbbslogo}{mmbbslogo}
 \logo{\pgfuseimage{mmbbslogo}}



% Delete this, if you do not want the table of contents to pop up at
% the beginning of each subsection:
%%\AtBeginSubsection[]
%{
%\begin{frame}<beamer>
%\frametitle{Vorgehensweise}
%\tableofcontents[currentsection,currentsubsection]
%\end{frame}
%}

% If you wish to uncover everything in a step-wise fashion, uncomment
% the following command:

%\beamerdefaultoverlayspecification{<+->}

\begin{document}

\begin{frame}
\titlepage
\end{frame}

\begin{frame}
\frametitle{Einleitung}
\footnotesize{\tableofcontents}
% You might wish to add the option [pausesections]
\end{frame}


\section{ISDN}
\subsection{Definition und Merkmale}
\begin{frame}
\begin{itemize}
\item Integrated Services Digital Network
\item Internationaler Standard f�r digitales Kommunikationsnetzwerk
\item Integration verschiedener Dienste z.B. Telefonie, Telefax, Daten�bertragung
\item Zwei Anschlussvarianten
\begin{itemize}
  \item Basisanschluss
  \item Anlagenanschluss
\end{itemize}
%  \item fictitious settings \& actions
%  \vskip0pt plus.3fill
%  \item scripted or based on spontaneous acting
%  \vskip0pt plus.3fill
%  \item being someone else than yourself
\end{itemize}
\end{frame}

\subsubsection*{Basisanschluss}
\begin{frame}
\frametitle{Basisanschluss}
\begin{itemize}
  \item Zwei Nutzkan�le (B-Kan�le) mit je 64 Kbit/s
  \begin{itemize}
    \item Telefonie, Telefax \ldots
  \end{itemize}
  \item Ein Steuerkanal (D-Kanal) mit 16 Kbit/s
  \begin{itemize}
    \item Steuerinformationen
  \end{itemize}
  \item Nutzung von Privatkunden und kleineren Betrieben
\end{itemize}
\end{frame}

\subsubsection*{Anlagenanschluss}
\begin{frame}
\frametitle{Anlagenanschluss}
\begin{itemize}
  \item Anschluss einer Telefonanlage
  \item Eine Grundrufnummer mit beliebiger Anzahl von Durchwahlen
  \item Nutzung von mittelst�ndischen bis gro�en Betrieben 
\end{itemize}
\end{frame}

\subsubsection*{Prim�rmultiplex-Anschluss}
\begin{frame}
\frametitle{Prim�rmultiplex-Anschluss}
\begin{itemize}
  \item Besondere Form des Anlagenanschlusses
  \item 30 B-Kan�le mit je 64 Kbit/s
  \item Ein D-Kanal mit 64 Kbit/s
  \item Ein Kanal f�r Synchronisation und Wartung mit 64 Kbit/s
\end{itemize}
\end{frame}

\subsection{Vergleich von ISDN- und Analog-Anschl�ssen}
\begin{frame}
\begin{small}
\begin{columns}[c]
\column{1.8in}
Analog-Anschl�sse:
	\begin{itemize}
      \item 1 Kanal
      \item 1 Rufnummer
      \item 56 Kbit/s
      \item Signalverst�rkung\\
      \item Basispreis: 17~\euro/Monat
    \end{itemize}
\column{1.8in}
ISDN-Anschl�sse:
	\begin{itemize}
      \item 2 Kan�le
      \item 1 - 10 Rufnummern
      \item 128 Kbit/s
      \item Signalregenerierung\\
      \item Basispreis: 17,90~\euro/Monat
    \end{itemize}
\end{columns}
\end{small}
\end{frame}

\subsection{Kosten von ISDN- und Analog-Anschluss}
\begin{frame}
	\begin{figure}[htb]
		\centering
		\begin{tikzpicture}
			\begin{axis}[
				xlabel=Zeit (s),
				ylabel=Kraft (N),
				width=7cm,
				height=3cm]
			\addplot[smooth,color=blue] 
				table[x=time,y=datadownsampled] {demo.table};
			\end{axis}
		\end{tikzpicture}
	\end{figure}
\end{frame}

\subsection{Merkmale einer ISDN-TK-Anlage}
\begin{frame}
	\begin{itemize}
      \item Punkt zu Mehrpunkt-Verbindung
      \item Eine Rufnummer mit beliebigen Durchwahlnummern
      \begin{itemize}
	      \item Rufumleitung
	      \item Weiterverbinden
	      \item zentrales Telefonbuch
	      \item Geb�hrenerfassung
	  \end{itemize}
      \item Kostenlose interne Gespr�che
      \item Konferenzschaltungen\\
      \item Aktueller Basispreis: 17,90 \euro/Monat 
    \end{itemize}
\end{frame}


\section{DSL}
\subsection{Definition und Merkmale}
\begin{frame}
\frametitle{DSL}
	\begin{itemize}
      \item Digital Subscriber Line
      \item �bertragungsstandard mit bis zu 200 MBit via Kupferkabel
      \item DSL ``standalone''
      \item DSL �ber POTS
      \item DSL �ber ISDN
    \end{itemize}
    % FREQUENZSPEKTRUM-BILD!
\end{frame}

\subsection{ADSL}
\begin{frame}
\frametitle{ADSL}
	\begin{itemize}
      \item Asymmetric Digital Subscriber Line
      \item Prim�r f�r Privatanwender
      \item Downstream h�her als Upstream
      \item ADSL
      \item ADSL2 \& ADSL2+
    \end{itemize}
    % T-DSL-Angebot
\end{frame}

\subsection{SDSL}
\begin{frame}
\frametitle{SDSL}
	\begin{itemize}
      \item Symmetric Digital Subscriber Line
      \item Prim�r f�r Businessanwender
      \item Downstream entspricht Upstream
      \item Kein Festnetzanschluss m�glich, deswegen als Datenanschluss bezeichnet
      \item Aktuell bis zu 20 MBit realisierbar
    \end{itemize}
\end{frame}


\subsection{Vergleich der Transferdauer bei ADSL und SDSL}
\begin{frame}
	\begin{itemize}
      \item 
    \end{itemize}
\end{frame}

\subsection{Kosten eines DSL-Anschlusses}

\begin{frame}[plain]
\frameheading{Kosten eines DSL-Anschlusses - ADSL}
\begin{figure}
\begin{center}
  \includegraphics[width=1\textwidth]{DSL}
\end{center}
\end{figure}
\end{frame}

\begin{frame}[plain]
\frameheading{Kosten eines DSL-Anschlusses - SDSL}
\begin{figure}
\begin{center}
  \includegraphics[width=1\textwidth]{SDSL}
\end{center}
\end{figure}
\end{frame}


\section{Allgemein}
\subsection{Downloadtransferdauer von Analog, ISDN und ADSL}
\begin{frame}
	\begin{itemize}
      \item 
    \end{itemize}
\end{frame}

\subsection{Kostenvergleich}
\begin{frame}
	\begin{itemize}
      \item 
    \end{itemize}
\end{frame}

\section[]{}

\begin{frame}
\frametitle{Quellen}
\begin{itemize}
  \begin{footnotesize}
  \item QSC AG
  \item Telekom/T-Home AG
  \item htp GmbH
  \item Net IT, Verlag Handwerk und Technik
  \item \href{http://www.elektronik-kompendium.de}{http://www.elektronik-kompendium.de}
  \item \href{http://www.a-enterprise.ch}{http://www.a-enterprise.ch}
  \item \href{http://www.triple-tec.de}{http://www.triple-tec.de}
  \item IT-Handbuch f�r Fachinformatiker, Verlag Galileo Computing
  \item Einf�hrung in die Informatik, Verlag Oldenburg
  \end{footnotesize}
\end{itemize}
  \vskip0pt plus.5fill
\begin{scriptsize}  \ldots mit \LaTeX erstellt\end{scriptsize}
\end{frame}

\section[]{}
\begin{frame}
\begin{center}
\Large Danke f�r eure\\Aufmerksamkeit und Kritik!\end{center}
\end{frame}


\end{document}
